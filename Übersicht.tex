\documentclass[border=10pt]{standalone}
\usepackage{smartdiagram} %für tikz
\usepackage[ngerman]{babel} %Sprache deutsch (für Silbentrennung)
\usepackage[utf8]{inputenc} %UTF-Code für Umlaute
\usepackage[T1]{fontenc} %Trennung von Wörtern mit Umlauten
\usepackage{lmodern} %Schriftart
%\usepackage{libertine}
\renewcommand*\familydefault{\sfdefault} %Serifenlose Schrift als Standard
\usepackage{microtype}
\usesmartdiagramlibrary{additions}
\usetikzlibrary{fit}
\usetikzlibrary{decorations.pathreplacing}

\tikzstyle{container} = [draw, rectangle, semithick, inner sep=0.3cm
]

%Aufzählungsstriche
\AtBeginDocument{
	\def\labelitemi{\normalfont\bfseries{--}}
}


\begin{document}
	%Stand: 9.11.2020
	
	\begin{tikzpicture}[
		every node/.style = {shape=rectangle, % is not necessary, default node's shape is rectangle
			rounded corners,
			%	draw, semithick,
			text width=7cm,
			align=center,
			node distance=0.1cm
		}
		]
		
		%Überschriften
		\node (Mitte)[]{};
		
		\node (Titel)[above= 1cm  of Mitte, text width=15cm,
		]{\large Wissenschaftliche Politikberatung\\
			\normalsize Eine Themensammlung \today };
		
		\node [below left = 1cm of Mitte](T1) {\textbf{(Mangelnde) Unabhängigeit} \\ 
			\begin{itemize}
				\item Transparenz/Wahrhaftigkeit
				\item konkrete Auftragsformulierung
				\item Offenlegung von Kooperationsstrukturen
		\end{itemize}};
		
		\node [right =  of T1] (T2) {\textbf{Datenverfügbarkeit} \\ 
			\begin{itemize}
				
				\item Datenschutz als Problem (?)
				\item Bevorzugung von ausländischen Themen
				\item Berücksichtigung verfügbarer Datensätze
				\item Internationale Angleichung des Rechtsrahmens (?)
				
		\end{itemize}}; 

	\node [right =  of T2] (T3) {\textbf{Ethische Maßstäbe} \\ 
		\begin{itemize}
			
			\item Transparenzpflichten
			\item Objektivität/Wahrhaftigkeit
			\item Keine unnötige Polarisierung
			\item Einbettung in den wissenschaftlichen Kontext
			\item Fairer Wettbewerb im Wissenschaftsbetrieb
			\item Auswahl vorrangig auf der Basis von Kompetenz
		
			
	\end{itemize}}; 	

	\node [right =  of T3] (T4) {\textbf{Anreize} \\ 
		\begin{itemize}
			
			\item Fehlende Wertschätzung der Beratung
			\item Fehlende Berücksichtigung der Ergebnisse
			\item Politische Anreize vs. wissenschaftliche Anreize
			\item 
			
	\end{itemize}}; 
	
		\node [right =  of T4] (T5) {\textbf{Kommunikationsprobleme} \\ 
		\begin{itemize}
			
			\item Unklare Auftragserteilung
			\item Unklare Absprachen zwischen Auftraggeber und "~nehmer
			\item Kommunikation der Ergebnisse
			
	\end{itemize}}; 
	
		
		
		
	\end{tikzpicture}
	
\end{document}