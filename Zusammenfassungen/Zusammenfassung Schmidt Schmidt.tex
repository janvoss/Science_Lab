\documentclass[border=10pt]{standalone}
\usepackage{smartdiagram}
\usepackage[ngerman]{babel}
\usepackage[utf8]{inputenc}
\usepackage[T1]{fontenc}
\usepackage{lmodern}
\renewcommand*\familydefault{\sfdefault}
\usepackage{microtype}
\usesmartdiagramlibrary{additions}
\usetikzlibrary{fit}
\usetikzlibrary{decorations.pathreplacing}

\AtBeginDocument{
	\def\labelitemi{\normalfont\bfseries{--}}
}

\begin{document}
	\begin{tikzpicture}[		every node/.style = {			shape=rectangle,			rounded corners,			text width=5cm,			align=center,			node distance=0.1cm		}	]
	
	\node (Kernpunkte)[text depth=.25ex]{\textbf{Kernpunkte}};
	
	\node (Forderungen)[text depth=.25ex, right = of Kernpunkte]{\textbf{Forderungen}};
	
	\node (Kommentar)[text depth=.25ex, right = of Forderungen]{\textbf{Kommentar}};
	
	%%%%%%%%%%%%%%%%%%%%%%%%%%%%%%%%%%%%%%%%%%%%%%%%%%%%%


	
	\node[below= of Kernpunkte, align=left]{
		\begin{itemize}
		\item Klimaschutz und Deindustrialisierung sind miteinander verbunden
		\item Ziel: Fossilwirtschaft durch Klimaschutz und Ressourceneffizienz ersetzen
		\item Entkoppelung von Wirtschaftswachstum und Ressourcenverbrauch notwendig
		\end{itemize}
	};
	
	\node[below= of Forderungen, align=left]{
		\begin{itemize}
		\item Verstärkte Investitionen in erneuerbare Energien, Energieeffizienz und Kreislaufwirtschaft
		\item Entwicklung und Einsatz von Technologien zur Reduktion von Treibhausgasemissionen
		\item Schaffung von Anreizen für Unternehmen und Konsumenten zur Reduktion von Treibhausgasemissionen
		\end{itemize}
	};
	
	\node[below= of Kommentar, align=left]{
		\begin{itemize}
		\item Der Artikel zeigt die Herausforderungen auf, die mit der Deindustrialisierung und Defossilierung verbunden sind
		\item Die Forderungen nach verstärkten Investitionen und Technologieentwicklung sind ein wichtiger Schritt, um diese Herausforderungen zu meistern
		\item Der Text gibt eine gute Übersicht über die wichtigsten Handlungsoptionen und ihre möglichen Auswirkungen
		\end{itemize}
	};
	
	
	
	%%%%%%%%%%%%%%%%%%%%%%%%%%%%%%%%%%%%%%%%%%%%%%%%%%%%%
	
	
	\draw [			transform canvas={yshift=-0.1cm}		] (Kernpunkte.south west) -- (Kommentar.south east);
	
	\path [draw=none] (Kernpunkte) -- (Kommentar) node [midway] (Mittelpunkt) {};	
	\node[above= 1cm of Mittelpunkt, text width=15cm]{\huge Schmidt/ Schmidt (2022): Defossilierung vorantreiben und Deindustrialisierung vermeiden: möglich, aber schwierig};
	
	\end{tikzpicture}
\end{document}
