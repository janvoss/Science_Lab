%% LyX 2.2.1 created this file.  For more info, see http://www.lyx.org/.
%% Do not edit unless you really know what you are doing.
\documentclass[11pt,a4paper,ngerman]{article}
\usepackage[T1]{fontenc}
\usepackage[utf8]{inputenc}
\usepackage{babel}
\usepackage{setspace}
%\usepackage{jurabib}[2004/01/25]
\onehalfspacing
%\usepackage[unicode=true,pdfusetitle,
% bookmarks=true,bookmarksnumbered=false,bookmarksopen=false,
 %breaklinks=false,pdfborder={0 0 0},pdfborderstyle={},backref=false,colorlinks=false]
 %{hyperref}

\makeatletter

%%%%%%%%%%%%%%%%%%%%%%%%%%%%%% LyX specific LaTeX commands.
%\pdfpageheight\paperheight
%\pdfpagewidth\paperwidth

%\providecommand{\LyX}{\texorpdfstring%
  %{L\kern-.1667em\lower.25em\hbox{Y}\kern-.125emX\@}
  %{LyX}}

%%%%%%%%%%%%%%%%%%%%%%%%%%%%%% User specified LaTeX commands.
\usepackage[a4paper]{geometry}
\usepackage[bottom, hang]{footmisc}
\usepackage{graphicx}
\usepackage{totpages}
\usepackage{libertine}
\usepackage{microtype}
\usepackage{blindtext}
\usepackage{hologo}
\usepackage{enumitem}
\usepackage{fancyhdr}
\usepackage{verbatim}
\pagestyle{fancy}
\lhead{}
\chead{}
\rhead{ \includegraphics[height=0.40in]{HFWU-logo_hell.jpg}}
\cfoot{\thepage /\ref{TotPages}}
\renewcommand{\headrulewidth}{0pt}

\fancypagestyle{ErsteSeite}{%
   \fancyhf{}%
   \rhead{\includegraphics[height=0.40in]{HFWU-logo_hell.jpg}\\ \smallskip{} Vertraulich! Entwurf\\ Stand: \today\\ Vo}
\cfoot{\thepage /\ref{TotPages}}
} 
\clubpenalty=10000
\widowpenalty=10000

\AtBeginDocument{
  \def\labelitemi{\normalfont\bfseries{--}}
}

\makeatother

\begin{document}

\title{Science Lab: Übungsfragen}

\date{Sommersemester 2024}

\author{Prof.\,Dr.\,Jan S.\,Voßwinkel}

\maketitle
%\thispagestyle{ErsteSeite} %Dadurch erscheinen auf der ersten Seite Logo, Stand: Datum, Vertraulichkeitshinweis und Kürzel.
\thispagestyle{fancy} %Dadurch erscheint auf der ersten Seite das Logo.

\section{Wissenschaftstheorie}
	\begin{enumerate}
		\item Was ist wissenschaftliches Wissen?
		\item Worauf beruht Wissen?
		\item Was zeichnet gute Begründungen aus?
		\item Wie verhält es sich mit der \glqq Wahrheit\grqq\ von Meinungen?
		\item Was ist der Unterschied zwischen Wissen und Meinung?
		\item Was ist mit der Forderung gemeint, dass Wissen \glqq irritationsfest\grqq\ sein soll?
		\item Was heißt es, dass Wissen irrtumssensibel sein soll?
		\item Was ist der Unterschied zwischen singulären Tatsachen und allgemeinen Tatsachen?
		\item Was ist damit gemeint, dass die Begründung von Meinungen  im Rahmen der Wissenschaft kontextabhängig ist?
		\item Woran erkennt man \glqq Fake News\grqq?
		\item Was ist Wissenschaft nach Auf"|fassung des Bundesverfassungsgerichts?
		\item Was ist Objektivität?
		\item Was ist der Unterschied zwischen phänomenalem Wissen, praktischem Wissen und satzförmigem Wissen?
		\item Was spricht dafür, dass die Bewertung eines Sachverhaltes in eine Erklärung des Sachverhaltes eingebettet sein sollte?
	%	\item Erklären Sie die üblichen Schritte der ökonomischen Analyse und Praxis anhand des Schemas \glqq beobachten \(\rightarrow\)\ beschreiben \(\rightarrow\)\ erklären \(\rightarrow\)\ bewerten \(\rightarrow\)\ gestalten\grqq.
	%	\item Was meint \glqq kritischer Rationalismus\grqq?
		%\item Was gilt es bei Prognosen in die Zukunft zu beachten?
	
	\end{enumerate}
	

\section{Wissenschaftliches Arbeiten}
	\begin{enumerate}[resume]
		\item Warum muss man die Übernahme fremden Gedankengutes deutlich machen?
		\item Gemeinsamkeiten und Unterschiede zwischen wissenschaftlichem und journalistischem Arbeiten/Schreiben
		\item Wann verwendet man ein indirektes Zitat? Wann ein direktes?
		\item Drei Absätze basieren auf einer Quelle. Wie macht man die Übernahme fremder Gedanken kenntlich?
		\item Wie werden Literaturquellen im Literaturverzeichnis angeordnet?
		\item Welche Arten von Literatur gibt es in den Wirtschaftswissenschaften? (Buch, Sammelband, wissenschaftliche Zeitschrift, Working Paper)
		\item Welche Vor"~ und Nachteile sind mit diesen verschiedenen Formen verbunden?
		\item Welche bibliografischen Angaben benötigt man für
		\begin{enumerate}
			\item Fachaufsatz in wissenschaftlicher Zeitschrift
			\item Monografie
			\item Aufsatz in einem Sammelband?
			\item Ein Working Paper?
			\item Internetseite?
		\end{enumerate}
		\item Welche Informationen sollten in der Einleitung einer wissenschaftlichen Arbeit enthalten sein?
		\item Es gibt verschiedene Auf"|fassungen über den Inhalt eines Fazits. Was haben wir im Seminar zum Inhalt des Fazits gesagt?
		\item Darf man seine \glqq eigene Meinung\grqq\ in eine wissenschaftliche Arbeit einfließen lassen?
	\end{enumerate}

\begin{comment}


\section{Zeitgespräch}
\subsection{Brozus/Geden}
	\begin{enumerate}[resume]
		\item Was sind alternative Fakten?
		\item Was ist gemeint mit Evidenzrezeption und Evidenzproduktion?
		\item Was heißt Outsourcing?
		\item Was trug zur Politisierung der Wissenschaft bei?
		\item Was sind normative Vorannahmen?
		\end{enumerate}
 \subsection{\'{C}umurov\'{c}/Gropp}
 	\begin{enumerate}[resume]
 		\item Warum verschaffen sich bei Reformen Verlierer eher Gehör als Gewinner?
 		\item Was ist von der Kritik zu halten, die Ökonomik habe die Krise nicht vorhergesehen?
	\end{enumerate}

\subsection{Paqué}
\begin{enumerate}[resume]
	\item Was mach Demoskopen?
	\item Warum mehr Beteiligung statt mehr Beratung?
	\item Was heißt heuristische Vorläufigkeit?
\end{enumerate}

\end{comment}

\end{document}
